\subsection{Justificativa}\label{sec:justificativa}

A incorporação de modelos de linguagem generativa (LLMs) nos fluxos de escrita acadêmica torna pertinente investigar, com maior precisão, quais dimensões do desempenho textual de estudantes são efetivamente favorecidas e quais vulnerabilidades emergem. A literatura recente indica que o uso de IA pode acelerar ciclos de revisão, oferecer feedback imediato sobre coesão e correção linguística e apoiar a organização retórica do texto; por outro lado, há alertas sobre homogeneização estilística, redução da variação lexical e deslocamento do foco do conteúdo para a forma \cite{kobak2025llm,geng2024style,yeadon2024ai}. Tais achados, articulados à discussão editorial e ética sobre transparência de uso e responsabilidade autoral, sustentam a relevância do estudo ao situá-lo na interseção entre inovação pedagógica e integridade acadêmica \cite{thorp2023notauthor,nature2023groundrules,cope2023ai,wame2023recs,icmje2023ai}.

Para estudantes de graduação, a pertinência é direta. Em disciplinas com forte componente de produção textual, a possibilidade de obter explicações, exemplos de estruturação e sugestões de reescrita pode apoiar o desenvolvimento de competências de planejamento e revisão — desde que conduzido com mediação docente e objetivos de aprendizagem explícitos. Diretrizes internacionais sugerem que o uso formativo de IA, com prompts orientados, rubricas claras e reflexão metacognitiva, tem potencial para ampliar a autorregulação da escrita \cite{unesco2023diretrizes,educause2023gai}. Ao mesmo tempo, a literatura aponta riscos pedagógicos: dependência de recomendações superficiais, aceitação acrítica de sugestões e dificuldades em distinguir entre ganhos de fluência (microestrutura) e ganhos de conteúdo (profundidade analítica) \cite{geng2024style,yeadon2024ai}. Em contextos de maior desigualdade de acesso a acompanhamento individualizado, ferramentas generativas podem reduzir assimetrias de feedback; porém, sem orientação, também podem cristalizar padrões genéricos pouco sensíveis às especificidades disciplinares.

No eixo da integridade e da ética, organismos e editorias convergem para princípios: IA não é autora; o uso deve ser declarado; responsabilidade humana permanece central na precisão factual, na originalidade e na qualidade metodológica \cite{thorp2023notauthor,cope2023ai,wame2023recs,icmje2023ai}. Além disso, recomenda-se documentação das interações com sistemas e revisão humana substantiva, bem como cautela com alucinações e vieses. Políticas editoriais estabelecem balizas para submissões envolvendo IA, reforçando práticas de transparência e checagem \cite{nature2023groundrules}. Tais orientações são particularmente relevantes em ambientes formativos, nos quais é preciso conjugar aprendizado de práticas acadêmicas com salvaguardas contra plágio, fabricação de referências e atribuição indevida de crédito.

Em relação ao escopo específico desta pesquisa, a leitura da Introdução (seção \ref{sec:introducao}) e da pergunta central (seção \ref{sec:pergunta-central}) aponta lacunas que justificam a investigação: a) mensuração diferenciada de efeitos em língua portuguesa, considerando critérios de qualidade argumentativa, clareza e precisão conceitual; b) impactos do uso assistido na autonomia escritora e na transferência de aprendizados para tarefas sem suporte de IA; c) condições de mediação docente e de políticas institucionais que favorecem usos responsáveis (transparência, documentação e avaliação justa). Ao focalizar estudantes universitários, o estudo endereça uma etapa crítica de formação na qual hábitos de escrita e referenciação se consolidam \cite{unesco2023diretrizes}.

Espera-se, como contribuição científica, oferecer evidências situadas sobre quando e como a IA melhora processos e produtos de escrita, distinguindo ganhos de forma e de conteúdo; e, como contribuição prática, propor diretrizes pedagógicas acionáveis para desenho de atividades, rubricas e protocolos de transparência, úteis a docentes e instituições. Originalmente, o estudo combina análise de qualidade textual com variáveis de processo (planejamento, revisão, uso de feedback automatizado), alinhando inovação didático-metodológica a salvaguardas de integridade. Ao ancorar-se na literatura recente \cite{kobak2025llm,geng2024style,yeadon2024ai,thorp2023notauthor,nature2023groundrules,cope2023ai,wame2023recs,icmje2023ai,unesco2023diretrizes,educause2023gai,freire1987} e conectar-se explicitamente à questão de pesquisa apresentada, a justificativa consolida a relevância acadêmica, pedagógica e ética da investigação, sem repetir conteúdos já desenvolvidos na Introdução.
