\section{Pergunta Central}\label{sec:pergunta-central}
\noindent\textbf{Questão de pesquisa:} Como a inteligência artificial influencia a escrita acadêmica de estudantes universitários?

A difusão de modelos de linguagem generativa (LLMs) reconfigura práticas de escrita e revisão em contextos acadêmicos, suscitando debates sobre qualidade textual, autoria e integridade. Evidências recentes mostram traços estilísticos e vocabulários característicos de textos assistidos por IA em publicações científicas, indicando mudanças detectáveis na produção acadêmica \cite{kobak2025llm}. Em paralelo, investigações têm analisado se e como o uso de IA altera o estilo e a densidade textual de pesquisadores, sugerindo efeitos mensuráveis na forma de escrever e apresentar argumentos \citeonline{geng2024style}.

No âmbito formativo, estudos comparativos entre redações avaliadas por especialistas e textos gerados ou apoiados por IA em contextos educacionais apontam para desafios de avaliação, critérios de originalidade e desenvolvimento de competências autorais \cite{yeadon2024ai}. Diante desse cenário, esta pesquisa pergunta em que medida a IA, como apoio ao processo de escrita, impacta a autonomia, a clareza argumentativa e a integridade acadêmica de estudantes universitários, considerando benefícios (feedback rápido, organização de ideias) e riscos (dependência excessiva, homogeneização estilística). A resposta informará diretrizes pedagógicas para o uso responsável de IA na escrita acadêmica no ensino superior.