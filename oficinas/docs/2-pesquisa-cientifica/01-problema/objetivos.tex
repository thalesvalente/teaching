\subsection{Objetivos}\label{sec:objetivos}

\subsubsection{Objetivo Geral}
Avaliar a influência da inteligência artificial na escrita acadêmica de estudantes universitários.

\subsubsection{Objetivos Específicos}
\begin{enumerate}
  \item Mensurar diferenças em métricas textuais (p.ex., riqueza lexical, coesão e organização retórica) entre textos produzidos com e sem apoio de IA. % Métodos: ver Seção Metodologia — delineamento comparativo e análise textual
  \item Descrever padrões de uso da IA no processo de escrita (planejamento, rascunho, revisão e reescrita) e sua relação percebida com a autonomia do estudante. % Métodos: questionário/diário de processo; análise qualitativa
  \item Analisar efeitos do uso de IA sobre a integridade acadêmica, considerando transparência de uso, acurácia de referências e aderência a diretrizes editoriais. % Métodos: checklist de conformidade e auditoria de referências
  \item Avaliar a associação entre políticas/orientações institucionais e a qualidade/ética do uso de IA em tarefas avaliativas. % Métodos: mapeamento de políticas + correlação com indicadores de qualidade
  \item Comparar a clareza argumentativa e a estrutura dos textos (introdução, desenvolvimento, conclusão) em amostras com diferentes níveis de apoio de IA. % Métodos: rubrica analítica baseada em critérios de IMRaD/ABNT
  \item Explorar fatores moderadores (curso/área, experiência prévia com IA, proficiência em escrita) nos efeitos observados. % Métodos: análise de subgrupos e testes de interação
\end{enumerate}

% Observações metodológicas (fundamentação para formulação de objetivos; autor-data):
% - Objetivos situados na Introdução e expressos com verbos de ação, conforme ABNT NBR 6022 e NBR 14724.\cite{abnt6022,abnt14724}
% - Citações e remissões no corpo do texto devem seguir ABNT NBR 10520.\cite{abnt10520}
% - Estrutura lógica do manuscrito alinhada ao IMRaD e recomendações editoriais (ICMJE).\cite{icmje2025recs}
% - Conformidade tipográfica e de elementos pré/textuais com abnTeX2.\cite{abntex2manual}
